This chapter gives an introduction to program analysis and to the working of
parallel programs and relaxed memory models.
We will also present concepts which we will use later in the work and introduce programming languages used in the examples.

%%%%%%%%%%%%%%%%%%%%%%%%%%%%%%%%%%%%%%%%%%%%%%%%%%%%%%%%%%%%%%%%%%%%%%%%%%%%%%%
\section{Program Analysis} %%%%%%%%%%%%%%%%%%%%%%%%%%%%%%%%%%%%%%%%%%%%%%%%%%%

In this work, we will not be concerned with simple program analysis techniques like type-checkers and linters.
Instead, we will focus on techniques based on formal verification, that is techniques which can provide some formally described guarantees about the program if the succeed.
These techniques usually require a program and some specification which the program should adhere to, and can check if that is really the case.
We will also focus mostly on techniques which can be directly applied to programs written in some mainstream programming language (i.e., techniques which do not require special verification-oriented languages as their inputs).

Program analysis techniques can be broadly divided into two different areas, \emph{automatic techniques} which, provided a program and its specification, should produce a result automatically without further assistance, and \emph{human-asisted techniques} which require a substantial human effort during the analysis.
Examples of the first area are symbolic execution and various model checking techniques, while theorem proves are example of the latter kind.
As we focus on techniques which should be usable by programmers without a substantial background in logic we will focus primarily on the automatic tools.


\subsection{State Space}

In order to be able to define program properties which should be checked and to describe various analysis methods, we first need to define the \emph{state space of the program}.

\begin{definition}[State Space]
The \emph{state space} of a program is a directed multigraph with (optionally) labelled edges which describes all the ways in which the program can be executed.
The vertices of the state space multigraph are called \emph{states} (of the program).
Each state represents a certain point in the execution of the program -- it can be described by a snapshot of the program (its memory, program counters and stacks of all its threads, …).
States $v_1, v_2$ are connected by an edge in the state space if $v_2$ can be reached from $v_1$ by a single step of the program.
The edge can be optionally labelled, for example by the statements executed in each step, or by a selected actions specified by the program or verification tool (such as \emph{error} label to indicate an error occurs on this edge or \emph{accepting} label which can be used in automata-based approach to LTL model checking).
The state space contains all the states of the program which are reachable from a certain initial state (an initial configuration of the program).
\end{definition}

The state space is usually given not as a graph, instead it is specified by its
implicit representation -- the program code, or possibly a function which
describes the initial states and how to get from one state to its successors.

In practice, the state space of a program can be very large or infinite and it
is useful to be able to consider only a representative part of the state space.

\begin{definition}[Reduced State Space]
A \emph{reduced state space} of a program is a subset of the state space in which states are subset of states of the original state space and edges connect states between which there is a directed path in the original state space such that all the internal states of this path are not contained in the reduced state space.\mnote{
For example, consider the following state space:\\
\begin{tikzpicture}[>=latex,>=stealth',auto,node distance=1em,semithick,initial text=,
      state/.style={inner sep=3pt, draw, circle}]
  \node[state, initial, initial above] (i) {};
  \node[state, below left = of i] (a) {};
  \node[state, below right = of i] (b) {};
  \node[state, below left = of a] (aa) {};
  \node[state, below right = of a] (ab) {};
  \node[state, below right = of b] (bb) {};
  \node[state, below right = of aa] (aab) {};
  \node[state, below right = of ab] (abb) {};
  \node[state, below right = of aab] (aabb) {};

  \path[->, shorten >=1pt]
	(i) edge (a)
	(i) edge (b)
	(a) edge (aa)
	(a) edge (ab)
    (b) edge (ab)
    (b) edge (bb)
    (aa) edge (aab)
    (ab) edge (aab)
    (ab) edge (abb)
    (bb) edge (abb)
    (aab) edge (aabb)
    (abb) edge (aabb)
  ;
\end{tikzpicture}

A corresponding reduced state space might be:\\
\begin{tikzpicture}[>=latex,>=stealth',auto,node distance=1em,semithick,initial text=,
      state/.style={inner sep=3pt, draw, circle},
      rem/.style={state, thin, dashed, draw=gray},
      ]
  \node[state, initial, initial above] (i) {};
  \node[state, below left = of i] (a) {};
  \node[rem, below right = of i] (b) {};
  \node[state, below left = of a] (aa) {};
  \node[rem, below right = of a] (ab) {};
  \node[rem, below right = of b] (bb) {};
  \node[state, below right = of aa] (aab) {};
  \node[rem, below right = of ab] (abb) {};
  \node[state, below right = of aab] (aabb) {};

  \path[->, shorten >=1pt]
	(i) edge (a)
	(i) edge[rem] (b)
	(a) edge (aa)
	(a) edge[rem] (ab)
    (b) edge[rem] (ab)
    (b) edge[rem] (bb)
    (aa) edge (aab)
    (ab) edge[rem] (aab)
    (ab) edge[rem] (abb)
    (bb) edge[rem] (abb)
    (aab) edge (aabb)
    (abb) edge[rem] (aabb)
  ;
\end{tikzpicture}

Another possibility is:\\
\begin{tikzpicture}[>=latex,>=stealth',auto,node distance=1em,semithick,initial text=,
      state/.style={inner sep=3pt, draw, circle},
      rem/.style={state, thin, dashed, draw=gray},
      ]
  \node[state, initial, initial above] (i) {};
  \node[rem, below left = of i] (a) {};
  \node[rem, below right = of i] (b) {};
  \node[state, below left = of a] (aa) {};
  \node[rem, below right = of a] (ab) {};
  \node[state, below right = of b] (bb) {};
  \node[rem, below right = of aa] (aab) {};
  \node[rem, below right = of ab] (abb) {};
  \node[state, below right = of aab] (aabb) {};

  \path[->, shorten >=1pt]
	(i) edge[rem] (a)
	(i) edge[rem] (b)
	(a) edge[rem] (aa)
	(a) edge[rem] (ab)
    (b) edge[rem] (ab)
    (b) edge[rem] (bb)
    (aa) edge[rem] (aab)
    (ab) edge[rem] (aab)
    (ab) edge[rem] (abb)
    (bb) edge[rem] (abb)
    (aab) edge[rem] (aabb)
    (abb) edge[rem] (aabb)
  ;
  \path[->, shorten >=1pt]
    (i) edge (aa)
    (i) edge (bb)
    (aa) edge (aabb)
    (bb) edge (aabb)
  ;
\end{tikzpicture}
}
\end{definition}

There are several techniques which can be used to construct reduced state space in such a way that a given property (or a class of properties) holds in the reduced state space if and only if it holds in the original state space.
For example partial order reduction~\cite{Peled1993}, dynamic partial order reduction~\cite{Flanagan2005dpor} or $\tau+$ reduction~\cite{RBB13}.

Finally, we will need the following state-space-related definitions.

\begin{definition}[Finite State Space]
We say that a program has a \emph{finite state space} if the number of vertices and edges in the state space is multigraph is finite.
Otherwise, we say the program has an \emph{infinite state space}.
\end{definition}

\begin{definition}[Run]
A \emph{run} in the state space is a (possibly infinite) path in the state space (or reduced state space) which starts in the initial state.
That is, a run is a sequence of states $\sigma = s_0, s_1, …$ such that $s_0$ is the initial state and for each consecutive pair of states $s_i, s_{i+1}$ there is an edge from $s_i$ to $s_{i + 1}$ in the corresponding (reduced) state space.
Please note that states can be repeated in a run.
\end{definition}


\subsection{Program Properties}

In order to be able to find errors in programs, an analysis tool has to have
some specification of program correctness -- a \emph{property} which should be
checked.
In this section, we will introduce several categories of program properties
together with examples of particular properties.

\paragraph{Safety Properties}
%
are properties which can be checked locally -- for
each state of the program, we are able to determine if the state satisfies or
violates given property.
Therefore, to conclude the program is error-free under given safety property it
suffices to show that there is no state which violates given safety property.

Examples of safety property include \emph{memory safety} (stating that every
time we access a certain part of memory, this memory is in fact allocated and
accessible to the program), \emph{assertion safety} (stating that a program
does not violate any of the assertions in the source code), and
\emph{control-flow definedness} (stating that every time a conditional jump is
performed, the values the jump is based on must have well-defined value).
Some program properties can be checked locally only if sufficient information
is kept in the program state.
For example checking \emph{absence of memory leaks} requires that it is
possible to enumerate all allocated objects and all objects to which there are
usable pointers.

\paragraph{Temporal properties}
%
Not all properties can be directly described as safety properties.
For example, we can have a property stating ,,every time a button is pressed, the elevator will eventually drive to the floor the button was pressed on'' -- such a property cannot be checked directly in one state of the program, but instead needs to be checked on a \emph{run} of the program.

Various temporal logics can be used to describe such properties, for example
LTL~\cite{todo}, CTL~\cite{todo}, CTL*~\cite{todo} or
$\mu$-calculus~\cite{todo}.

\paragraph{Termination Properties}
%
A distinguished form of temporal properties are termination properties, which are properties which allow us to specify that a program must (or must not) terminate, or that some of its part must (or must not) terminate.

\subsection{Hardware and Platform Related Considerations}

When a program is executed on a given hardware, some of its behaviour might
depend on the concrete features of the hardware.
Therefore, in order for the program analysis to be usable, one must know what
hardware model(s) the program analysis tool adheres to and what is their
relation to the real hardware the program will be executed on.
In practice, it is often hard to prove that a program will be correct on
\emph{any} hardware on which it will be possible to compile it.

For example, the size (and therefore precision) of the standard C/C++ type
\cpp{int} might depend on the hardware the program is using: on an 8-bit or
16-bit embedded microprocessor, \cpp{int} will likely be a 16-bit type with a
maximum value of $2^{15} - 1$, while on a common 64-bit (or 32-bit) computer it
will be a 32-bit number with maximum value of $2^{31}-1$.
This means a code which causes a bug due to integral overflow on a 16-bit
embedded microprocessor might be correct on a 64-bit machine.

Another example concerns alignment of values in the memory -- often it is
possible to address single bytes in the memory, but certain data types (which
are larger that a byte) are only allowed to start on addresses divisible by a
given \emph{alignment}.
For example, an alignment of a 32-bit \cpp{int} type might be 4 bytes (32
bits).
Depending on the hardware platform, reading and writing unaligned values might
work (e.g., on x86 and x86-64), might trigger an error (e.g., on certain ARM
processors \cite{??}) or the address might be silently rounded to the nearest
aligned address (e.g., on some embedded ARM processors \cite{??}).
Therefore, behaviour of the same program with an unaligned read can differ
drastically depending on the platform on which it is executed.

Furthermore, some features of the program are dictated not only by the
hardware, but also by the operating system (or more generally the platform) for
which it is compiled.
For example, the calling conventions of C programs compiled on Windows and on
Linux differ even on the same hardware.
Similarly sizes of data types might depend on the platform -- on Windows
\cpp{long} is 32 bits long even on 64-bit systems (and therefore cannot be used
to hold values of pointers), while it is 64 bits long on 64-bit Linux.


%%%%%%%%%%%%%%%%%%%%%%%%%%%%%%%%%%%%%%%%%%%%%%%%%%%%%%%%%%%%%%%%%%%%%%%%%%%%%%%%
\section{Parallelism \& Threading Model} %%%%%%%%%%%%%%%%%%%%%%%%%%%%%%%%%%%%%%%

Parallelism is an important part in programming high-performance software which
can fully take advantage of the current hardware.
However, parallelism comes with additional problems not present in development
of sequential software, namely it is significantly harder to create correct
parallel software compared to development of sequential software, and it is also
hard to create parallel software which scales well with the number of processors
(or processor cores) available.
Furthermore, improving scalability usually makes the problem of correctness even
harder -- improving scalability usually requires more more fine-grained (and
therefore error-prone) synchronization.
For example, fine-grained locking or lock-free programming can lead to
high-performance code, but seeing that the code uses synchronization properly
is significantly harder than for code which uses a few locks with well defined
regions parts of the shared memory they guard.

Furthermore, relaxed memory behaviour comes into play once shared variables are
not only accessed in critical sections.
As processor manufactures strive to increase speed of processors, they have
introduced various optimizations into the memory infrastructure to avoid
waiting for the relatively slow main memory.
These optimizations can be (and in most processors are) visible to a parallel
program and are manifested in relaxed memory behaviour.
For example, on x86 processors (which are used in most laptop, desktop and
server computers) a memory store can be delayed and appear, from the point of
view of the other threads, later than a subsequent load.
Some processors (for example POWER and higher-performance ARM processors) allow
even more reordering, for example reordering of independent loads.

In this section, we describe the basic model of parallelism we assume for our
programs and follow some of its implications.
We also outline basics of relaxed memory behaviour and memory models
which describe it.

\subsection{Basic Threading Model}

We assume that a program consists of one or more threads which interact using a
shared memory.\mnote{Other communication methods can be emulated using shared
memory. Furthermore, many programming languages, including C++, have no native
support for other communication schemes.}
We also assume that execution of a thread can be interrupted at any point by a
\emph{thread scheduler} (of the operating system on which the program runs) and
will be later eventually resumed (unless the program exits in the meantime or
the thread is \emph{blocked} infinitely).
A thread can be \emph{blocked} if it waits for a resource provided by the
operating system, for example availability of a lock or input from an input
device.
If a thread cannot be resumed because it is blocked and does never cease to be
blocked we say there is a \emph{deadlock}.

\begin{definition}[Dealock, Partial Dealock, Global Deadlock]\label{def:deadlock}
    A \emph{deadlock} happens if a scheduler never allows the thread to run
    because it waits for a resource which never becomes available.

    Sometimes, when we need to distinguish between all the threads being
    blocked and only some of them being blocked, we will use the notion of
    \emph{partial deadlock} to signify a situation in a program where some
    threads are blocked (in deadlock) but other can still proceed and
    \emph{global deadlock} for a deadlock which blocks all threads.
\end{definition}

It is important to note that our definition of deadlock does not include
\emph{busy waiting} -- i.e., a thread is executing a loop in which it tests
that some event occurred.
Busy waiting can be used for example to implement exclusive sections without
operating system support.

\begin{definition}[Livelock]\label{def:livelock}
    A \emph{livelock} is a situation in which a thread is allowed to run (by
    the scheduler), but does not proceed in any meaningful way because it
    executes a loop which is waiting for an event which ever happens.
\end{definition}

\paragraph{Interleaving of Threads, Sequential Consistency}

In practice program threads can run concurrently, i.e., multiple cores of the
processor can execute different threads at the same instant in time.
Furthermore, the scheduler can map threads to processor cores arbitrarily and
it can even change the core which executes a given thread.
It is not practical (and not necessary) to simulate this behaviour when
performing analysis of a parallel program.
Instead, we consider that threads are interleaved (\emph{interleaving
semantics}) or we consider execution based on some relaxed memory model.

\begin{definition}[Interleaving Semantics of Threads]
    With the \emph{interleaving semantics} we assume that all possible
    executions of parallel program can be obtained by interleaving.
    That is, at any point in execution of the program, we consider which threads
    are allowed to run and explore all possible selections.
    After a thread performs an action, we again consider all possible actions
    of all threads and so on.
\end{definition}



\subsection{Relaxed Memory Models}

A memory model describes behaviour of memory operations on a given platform.
When considering relaxed memory models, it is also useful to consider memory
model which is not relaxed and corresponds to the interleaving semantics of
threads.
This is the \emph{sequential consistency}.

\begin{definition}[Sequnetial Consistency]
    \emph{Sequential consistency} is the memory model which corresponds to the
    interleaving semantics of threads.
    Under sequential consistency, all effects of memory manipulating
    instructions are visible to all threads immediately and no instruction
    reordering is observable.
\end{definition}

Sequential consistency's operational semantics is given by a machine has no instruction reordering and in which all memory operations are atomic and access the memory directly, without any caches.
Sequential consistency is the strongest memory model (it is not relaxed at all).
Therefore, any behaviour observable under sequential consistency will also be possible under a relaxed memory model, but a relaxed memory model can exhibit additional behaviour.

In practice, sequential consistency is not sufficient with modern processors
that exhibit relaxed memory behaviour.
The relaxed behavior of processors arises from optimizations in cache
consistency protocols and observable effects of instructions reordering and
speculation.
The effect of this behavior is that memory-manipulating instructions can appear
to be executed in a different order than the order in which they appear in a
thread's code, and their effect can even appear to be in different order on
different threads.
For efficiency reasons, most modern processors (except for simple ones in
embedded microcontroller and low-cost mobile devices) exhibit relaxed behavior.
The extent of this relaxation is dependent on the processor architecture (e.g.,
x86, ARM, POWER) but also on the concrete processor model.
To make matters worse, the actual behavior of the processor is often not
precisely described by the processor vendor~\mcite{x86tso}.
To abstract from the details of particular processor models, \emph{relaxed
memory models} are used to describe (often formally) behavior of a given
processor architecture.
Examples of relaxed memory models of modern processors are the memory model of
x86 and x86-64 CPUs described formally as \xtso~\mcite{x86tso} and the multiple
variants of POWER~\mcite{Sarkar2011,MadorHaim2012} and
ARM~\mcite{Alglave2014,Flur2016,Pulte2017} memory models.

For the description of a memory model, it is sufficient to consider operations
which affect the memory.
These operations include loads (reading of data from the memory to a register
in the processor), stores (writing of data from a register to the memory),
memory barriers (which constrain memory relaxation), and \emph{atomic compound
operations} (read-modify-write operations and compare-and-swap operation).
Compound non-atomic instruction exist on some architectures (e.g., the
\texttt{add} instruction in x86 which can have one memory operand), but they
might be rewritten to equivalent sequence of load to register, register
modification, and store and therefore need not be considered.

\subsection{The \xtso Memory Model}\label{chap:prelim:xtso}

\TODO{Převzato z …}

The \xtso memory model is a formal description of the memory model used in x86
and x86-64 processors (manufactured by both Intel and AMD).
It is one of the strongest relaxed memory models -- it is relaxed compared to
sequential consistency, but not nearly as relaxed as some of the other common
memory models such as memory models of ARM and POWER processors.
The \xtso is very similar to the SPARC Total Store Order (TSO) memory
model~\mcite{SPARC94}.
It does not reorder stores with each other, and it also does not reorder loads
with other loads.
The only relaxation allowed by \xtso is that a store can appear to be executed
later than an independent load which succeeds it in a thread.
The memory model does not give any limit on how long a store can be delayed.
An example of non-intuitive execution of a simple program under \xtso can be
found in \autoref{fig:xtso}.

\begin{figure}[tp] % fig:xtso
    \begin{minipage}[c]{0.22\textwidth}
      \begin{cppcode}
        int x = 0;
        int y = 0;
        void t0() {
          y = 1;
          int a = x;
          int c = y;
        }
        void t1() {
          x = 1;
          int b = y;
          int d = x;
        }
      \end{cppcode}
    \end{minipage}
    %
    \hfill
    %
    \begin{minipage}[c]{0.77\textwidth}
    \begin{center}
    \noindent
    Is $a = 0 \land b = 0$ reachable?\\[2.5ex]
    \medskip
    \begin{tikzpicture}[ ->, >=stealth', shorten >=1pt, auto, node distance=3cm
                       , semithick
                       , scale=0.5
                       ]

      \draw [-] (-10,0) rectangle (-7,-5);
      \draw [-] (-10,-1) -- (-7,-1)
                (-10,-2) -- (-7,-2)
                (-10,-3) -- (-7,-3)
                (-10,-4) -- (-7,-4);
      \draw [-] (-9,0) -- (-9,-5);
      \node () [] at (-8.5,0.5) {shared memory};
      \node () [anchor=west] at (-10,-2.5)  {\texttt{\color{blue}x}};
      \node () [anchor=west] at (-9,-2.5) {\texttt{\color{blue}0}};

      \node () [anchor=west] at (-10,-3.5)  {\texttt{\color{blue}y}};
      \node () [anchor=west] at (-9,-3.5)  {\texttt{\color{blue}0}};

      \node () [anchor=center] at (-2.5,-3.5) {store buffer};
      \draw [-] (-4.5,-4) rectangle (-0.5,-5);
      \draw [-] (-2.5,-4) -- (-2.5,-5);

      \node () [anchor=center] at (3.5,-3.5) {store buffer};
      \draw [-] (1.5,-4) rectangle (5.5,-5);
      \draw [-] (3.5,-4) -- (3.5,-5);

      \node () [anchor=west] at (-4.5,-4.5)  {\texttt{\color{red}y}};
      \node () [anchor=west] at (-2.5,-4.5)  {\texttt{\color{red}1}};

      \node () [anchor=west] at (1.5,-4.5)  {\texttt{\color{red}x}};
      \node () [anchor=west] at (3.5,-4.5)  {\texttt{\color{red}1}};

      \node () [anchor = west, xshift = -1em] at (-4.5, 0.5) {thread 0};
      \draw [->] (-4.5,0) -- (-4.5,-3);
      \node () [anchor=west] at (-4, -0.5) {\texttt{\color{red}y = 1;}};
      \node () [anchor=west] at (-4, -1.5) {\texttt{\color{blue}load x; \textrightarrow 0}};
      \node () [anchor=west] at (-4, -2.5) {\texttt{\color{frombuf}load y; \textrightarrow 1}};

      \node () [anchor = west, xshift = -1em] at (1.5, 0.5) {thread 1};
      \draw [->] (1.5,0) -- (1.5,-3);
      \node () [anchor=west] at (2, -0.5) {\texttt{\color{red}x = 1;}};
      \node () [anchor=west] at (2, -1.5) {\texttt{\color{blue}load y; \textrightarrow 0}};
      \node () [anchor=west] at (2, -2.5) {\texttt{\color{frombuf}load x; \textrightarrow 1}};

  \end{tikzpicture}
  \end{center}
  \end{minipage}

  \caption{
  A demonstration of the \xtso memory model.
  The thread 0 stores 1 to variable \texttt{y} and then loads variables \texttt{x} and \texttt{y}.
  The thread 1 stores 1 to \texttt{x} and then loads \texttt{y} and \texttt{x}.
  Intuitively, we would expect it to be impossible for $a = 0$ and $b = 0$ to both be true at the end of the execution, as there is no interleaving of thread actions which would produce such a result.
  However, under \xtso, the stores are cached in the store buffers (marked \textcolor{red}{red}).
  A load consults only shared memory and the store buffer of the given thread, which means it can load data from the memory and ignore newer values from the other thread (\textcolor{blue}{blue}).
  Therefore \texttt{a} and \texttt{b} will contain old values from the memory.
  On the other hand, \texttt{c} and \texttt{d} will contain local values from the store buffers (locally read values are marked \textcolor{frombuf}{green}).
  The figure depicts state of the memory and buffer after the code of both
  threads executed, but before the data was propagated from store buffers to
  the main memory.
  }

  \label{fig:xtso}
\end{figure}

The operational semantics of \xtso is described in~\mcite{x86tso}.
The corresponding machine has multiple hardware threads (or cores), each with
associated local store buffer, a shared memory subsystem, and a shared memory
lock.
Store buffers are first-in-first-out caches into which store entries are saved
before they are propagated to the shared memory.
Load instructions first attempt to read from the store buffer of the given
thread, and only if they are not successful, they read from the shared memory.
Store instructions push a new entry to the local store buffer.
Entries in the store buffer are not visible to threads other then the one
owning the store buffer.
Atomic instructions include various read-modify-write instructions, e.g. atomic
arithmetic operations (which take memory address and a constant),\mnote{These
  instructions have the \texttt{lock} prefix in the assembly, for example
  \texttt{lock xadd} for atomic addition.}
or compare-and-swap instruction.\mnote{\texttt{lock cmpxchg}}
All atomic instructions use the shared memory lock so that only one such
instruction can be executed at a given time, regardless of the number of
hardware threads in the machine.
Furthermore, atomic instructions flush the store buffer of their thread before
they release the lock.
This means that effects of atomic operations are immediately visible, i.e.,
atomics are sequentially consistent on \xtso.
On top of these instructions, \xtso has a full memory barrier (\texttt{mfence})
which flushes the store buffer of the thread that executed it.\mnote{There
are two more fence instructions in the x86 instruction set, but according
to~\cite{x86tso} they are not relevant to normal program execution.}

If a programmer wishes to recover sequential consistency on x86, they need to
ensure memory stores are propagated to the main memory before subsequent loads
execute.
This is most commonly done in practice by inserting a memory fence after each
store.
An alternative approach would be to store using atomic exchange instruction
(\texttt{lock xchg}) which can atomically swap value between a register and a
memory slot.

One of the specifics of x86 is that it can handle unaligned memory
operations.\mnote{Other architectures, for example ARM, require loaded
values to be aligned, usually so that the address is divisible by the value
size.}
While the \xtso paper does not give any specifics about handling unaligned and
mixed memory operations (e.g., writing a 64-bit value and then reading a 16-bit
value from inside it) it seems from our own experiments that such operations
are not only fully supported, but they are also correctly synchronized if
atomic instructions are used.
This is in agreement with the aforementioned operational semantics of \xtso in
which all the atomic operations share a single global lock.

\subsection{Other Hardware Memory Models}

\TODO{… už nepřevzato}

Older works on analysis of programs under relaxed memory models usually
consider the SPARC TSO, PSO and RMO memory models~\mcite{SPARC94} or the memory
model of ALPHA~\mcite{mckenney2010} processors.
However, these memory models are not very relevant any more, with the exception
of TSO, which is sometimes used interchangeably with the \xtso memory model, as
they are very similar (the difference is that \xtso precisely describes
behaviour of atomic compound operations of x86 processors, but some works use
TSO to stand for the memory model of x86 processors too).

Currently, apart from the \xtso memory model, mostly the memory models of
variants of POWER and ARM processors are relevant.
While these processors differ significantly in the area in which they are used
(POWER is used in high-performance servers while ARM is mostly used in mobile
devices), their memory models share some basic features.
Recently, there is also interest in the memory model of RISC-V architecture, which is also similar to memory models of POWER and ARM.
All of these types of processors exhibit more relaxed behaviour then \xtso,
for example it is possible to reorder a load after a store, or to reorder
stores or loads with one another (provided they are independent).
On POWER it can also happen that a sequence of operations executed by a single thread will be observed by different other threads in different orders.
An example of such behaviour can be seen in \autoref{fig:prelim:iriw}.

\begin{figure}[tp]
    \begin{cppcode}
        int x = 0, y = 0;
    \end{cppcode}
    \begin{multicols}{4}
        \begin{cppcode}
          void wrt0()
          {
            x = 1;
          }
        \end{cppcode}
        \columnbreak
        \begin{cppcode}
          void wrt1()
          {
            y = 1;
          }
        \end{cppcode}
        \columnbreak
        \begin{cppcode}
          void read0()
          {
            int x0 = x;
            int y0 = y;
          }
        \end{cppcode}
        \columnbreak
        \begin{cppcode}
          void read1()
          {
            int y1 = y;
            int x1 = x;
          }
        \end{cppcode}
    \end{multicols}
    \center{\emph{Is it possible that $x_0 = 1 \land y_0 = 0 \land x_1 = 0 \land y_1 = 1$?}}
    \caption{
        An independent readers of independent writers example commonly used to
        demonstrate some of the features of POWER memory models.
        We assume each of the functions is executed in a separate thread.
        Here, we are asking if it can happen that while the \texttt{read0}
        observes the new value of $x$ and old value of $y$ the \texttt{read1}
        will observe the old value of $x$ and new value of $y$.
        Such an observation would imply that the two modifications are visible
        to the two readers in different order.
        This behaviour can indeed be observed on POWER~\cite{TODO}, but not on
        ARM~\cite{Pulte2017}.
    }\label{fig:prelim:iriw}
\end{figure}

There are several version (or generations) of ARM and POWER processors, and
also multiple versions of memory models.
The memory model of ARMv7 is described in~\mcite{Alglave2014}.
The memory model of newer ARMv8 is described in~\mcite{Flur2016}, but it was
later revised and simplified in collaboration with ARM and formalized
in~\mcite{Pulte2017} and in the ARMv8 architecture description.
Later in~\mcite{Pulte2019} an operation model of ARMv8 concurrency equivalent
with the one presented in~\mcite{Pulte2017} and optimized for program analysis
was presented.
Interestingly, \mcite{Pulte2019} also presents a memory model of RISC-V
architecture which is similar to the revised ARMv8 memory model.

The basis of POWER memory model was described as an abstract machine
in~\mcite{Sarkar2011} and later extended in~\mcite{Sarkar2012} to cover atomic
compound operations.
A more detailed memory model of POWER is presented in~\mcite{Gray2015},
including for example behaviour of mixed-size memory operations.
Axiomatic description of POWER is provided by~\mcite{MadorHaim2012} and
\mcite{Alglave2010_fences}.
In~\mcite{Flur2017} the authors describe behaviour of POWER and ARMv8 in
presence of mixed-size memory operations.

\subsection{Memory Models of Programming Languages}

Not all programming languages have concurrency and memory model defined in the
language.
Indeed, C and C++ prior to the 2011 revisions of their respective standards did
not define concurrency and therefore majority of concurrent C code relies on
non-standard concurrency.
In absence of programming language support, the programmers wishing to use
concurrency have to rely on combination of library and compiler support which
provides (platform-specific) means to support concurrency.
For example, on Linux and most POSIX-compatible operating systems, the POSIX
threads library (\texttt{pthreads}) defines ways to launch threads, wait for
them, and synchronize their execution using various (blocking) synchronization
primitives such as mutexes and condition variables.
A compiler compatible with this library then guarantees that its optimizations
will not break this functionality, for example that it will not reorder
operations around the calls to synchronization functions.
The compiler also usually defines low-level synchronization primitives which correspond to common atomic compound operations.
For example, the GCC and clang compilers both define builtin functions such as \texttt{\_\_atomic\_compare\_exchange} and \texttt{\_\_atomic\_add\_fetch}.
These buitins allow semi-portable use of atomic operations, in the sense they are not specific to the concrete hardware platform,\mnote{As would be the case if inline assembly was used to invoke the atomic instruction directly.} but they are instead bound to the given compiler or a group of compilers (such as GCC and clang).

The downside of the library and compiler approach to concurrency is that it
makes it hard to write truly platform-independent code in the given language.
For this reason, many programming languages (eventually) provides concurrency
primitives and define a memory model guaranteed by the language.

\paragraph{C and C++}\mnote{\srcnote{Based on TEDI}}
%
The C++11~\mcite{isocpp11draft} and C11~\mcite{isoc11draft} standards
introduced support for threading and atomic operations to C++ and C.
The memory model of both languages is the same, but they differ in the
syntactic ways in which its features are used\mnote{For example, C++ defines
\cpp{atomic} template class for atomic variables, but C relies on the
\cpp{_Atomic} keyword, as it has no support for (templated) classes.
The C threading API is mostly similar to the API of POSIX threads, while the
C++ threading has a more modern API which uses classes.}.

The C++ memory model was revised in the subsequent standards, with the most
notable changes in the C++20~\mcite{cpp20}, including changes to the strongest sequentially-consistent version of atomic operations.
These latest changes mostly remedy problems in interaction between the
sequentially consistent operations and weaker memory orderings presented
in~\mcite{Lahav2017}.
The actual changes are described in~\mcite{P0668R4}.

Overall, the C++ memory model is complex and complicated by its intention to
allow high-performance on various existing or hypothetical future hardware
platforms.

The C++ memory model is not formalized in the C++11 standard.
An attempt to formalize it was given in \mcite{cppmemmod}, formalizing the
N3092 draft of the standard \cite{N3092}.
While this formalization precedes the final C++11 standard, it seems that there were no changes in the specification of atomic operations after N3092.
Nevertheless, there are some differences between the formalization and N3092
(which are justified in the paper).
The formalization was later revised in~\mcite{Lahav2017}, which led to the aforementioned revision of concurrency in C++20.

Atomic variables and operations play a central role in in the C++ memory model.
Atomic variables are variables of special types which define atomic operations
such as as loads, stores, atomic read-modify-write, and compare-exchange.
For any atomic operation, it is possible to specify the required memory
ordering: C/C++ allows not only sequentially consistent atomic operations, but
also weaker (low-level) atomic operations which allows implementation of
efficient parallel data structures in a platform-independent way.
An example of C++ code with leverages atomic variables is shown in \autoref{fig:prel:cppatomic0} and \autoref{fig:prel:cppatomic1}.

\begin{figure}[tp]
  \begin{cppcode}
    std::atomic<int> x;
    int y;
  \end{cppcode}
  \begin{multicols}{2}
    \begin{cppcode}
      void thr0() {
        x.fetch_add(1); // OK
        y++; // RACE
      }
    \end{cppcode}
    \columnbreak
    \begin{cppcode}
      void thr1() {
        x++; // also OK
        y++; // RACE
      }
    \end{cppcode}
  \end{multicols}
  \vspace{-4ex}
  \caption{A basic example of C++ atomics.
      The variable \texttt{x} is atomic, and therefore can be safely used in
      concurrent settings (the common modification operators for integral types
      are available and atomic).
      On the other hand, the variable \texttt{y} is not atomic and therefore
      incrementing it from two threads causes data race.
  }\label{fig:prel:cppatomic0}
\end{figure}

\begin{figure}[tp]
  \begin{cppcode}
    Data data;
    std::atomic< bool > ready;

    void thr0() {
      data.load_data();
      read.store( true, std::memory_order::release );
    }

    void thr1() {
      while ( !read.load( std::memory_order::relaxed ) )
      { }
      std::atomic_thread_fence( std::memory_order::acquire );
      data.do_work();
    }
  \end{cppcode}
  \vspace{-2ex}
  \caption{A simple example of use of low-level atomic API.
      Here \texttt{thr0} loads \texttt{data} somehow and then signals to
      \texttt{thr1} that it should be processed.
      The \texttt{data} variable itself is not atomic and presumably the type
      is not made for concurrent access.
      An atomic boolean is used to wait for \texttt{data} in \texttt{thr1} --
      once \texttt{data} is loaded the \texttt{ready} flag is set, using the
      \emph{release} memory ordering.
      The actual waiting in \texttt{thr1} uses the weakest \emph{relaxed}
      ordering which does not cause any synchronization (it just ensures
      concurrent changes to the variable itself are consistent), but once the
      wait ends a fence is executed.
      The fence uses \emph{acquire} ordering and therefore completes
      \emph{release-acquire} ordering between \texttt{thr0} and \texttt{thr1}
      -- anything written by \texttt{thr0} before the \emph{release} store is
      available to \texttt{thr1} after the \emph{acquire} fence.
      An interesting property of this example is that it compiles with no extra
      synchronization or atomic instructions on x86 (atomic only affect
      compiler optimizations there), but uses synchronization on more relaxed
      platforms.
  }\label{fig:prel:cppatomic1}
\end{figure}

A notable feature of the C++ memory model is that any program which contains a
data race on a non-atomic variable\mnote{Data race is defined as two accesses
to the same non-atomic variable, at least one of them being a write, which are
not synchronized so that they cannot happen concurrently.} has undefined
behaviour.
This means that synchronization is possible only by atomic variables and
concurrency primitives such as mutexes and condition variables.

\paragraph{Java}\mnote{\srcnote{TEDI}}
%
The Java memory model is rather different from the C/C++11 one.
Its primary goal is to ensure that programs which cannot observe data races
under sequential consistency will execute as if running under sequential
consistency (the data race free guarantee)~\mcite{javamm_popl_Manson2005}.
The primary means of synchronization in Java are mutexes (called monitors in Java), synchronized sections of code (which use monitors internally), and volatile variables, which roughly correspond to sequentially consistent atomics in C++11.

Furthermore, as Java strives to be memory safe, it also defines behaviour of programs with data races.
This behaviour is rather peculiar, as it is primarily concerned with prohibiting out-of-thin-air values -- values which, informally speaking, depend cyclically on themselves.
These values are primarily prohibited to avoid forging pointers to invalid memory or memory which should be otherwise inaccessible to a given thread \mcite{javamm_popl_Manson2005}.

\paragraph{LLVM}\mnote{\srcnote{TEDI}}
%
The LLVM Intermediate Representation has a memory model derived from the C++
memory model, with the difference that it lacks release-consume ordering and
offers additional \emph{unordered} ordering which does not guarantee atomicity
but makes results of data races defined \mcite{llvm:langref}.
The \emph{unordered} operations are intended to match semantics of the Java
memory model for shared variables.

\section{Approaches to Multi-Threaded Software}

At the highest level, multiple threads of a program can either use
\emph{message passing}, or \emph{shared memory} to communicate.\mnote{This can
be generalized to multiple programs which cooperate, but the distinction is
mostly technical.  With multiple programs, message passing is used more often,
especially in distributed computing.}
While message passing can be done in any reasonably strong programming language
with support for concurrency, it is often used in programming languages which
adopt it as a main way of implementing concurrent or distributed programs, for
example Erlang or Go.
In this work, we are focusing on C++\mnote{And by extension to other
general-purpose programming languages with concurrency support, e.g. Java,
C\#.} which has no built-in support for message-passing concurrency, therefore
we will focus on shared memory concurrency.
In C++ and similar languages, message passing can be implemented as an
abstraction over the shared memory.

Synchronization plays a crucial role in shared memory concurrency -- concurrent
unsynchronized access of two or more threads to the same memory location, with
at least one of the threads writing to it will cause data race which can lead
to corruption of data and program malfunction.
Lock-based synchronization (critical sections) is often used as it is
reasonably simple to understand it and is often supported by the programming
language itself.\mnote{In C/C++ since the 2011 standards.}
Furthermore, locks usually use operating system primitives to block threads
which are waiting while other thread executes the critical section, which can
increase performance of systems with more threads then available processors --
the waiting threads are completely inactive and other threads can be executed
in the meanwhile.

On the other hand, synchronization can degrade performance of concurrent
software.
In the extreme case when all interesting work is done in a single critical
section executed by many threads, there will be no gain of concurrency and only
the overhead of running and synchronization of threads.
For this reason, critical sections should be as short as safely possible and
different areas of the shared memory should be protected by different locks so
that the program can access different areas of the shared memory concurrently.
This inherently increases complexity of synchronization and risks of data races
or deadlocks -- indeed deadlocks are easy to avoid with one lock, but with
multiple locks care must be taken if more then one lock is held at one time,
which is a common situation.

To further improve performance, it is sometimes useful to avoid
operating-system-assisted synchronization and use atomic operations instead.
These operations can be used to guarantee synchronization over a word-sized
location of the memory.\mnote{The size of the atomically-accessed location is
usually the same as the size of pointer on the given platform (e.g., 8 bytes on
64bit platforms), or sometimes twice the size of the pointer (e.g. 16 bytes on
newer x86-64 processors).}
Atomic operations are used to implement \emph{lock-free} algorithms and data
structures which do not require critical sections at all.
In this case extra care must be taken to the order of operations which cannot
be performed atomically due to the size limit of atomic instructions and the
programmer must carefully evaluate the impact of memory model for their
algorithm.
An example of (a fragment of) lock-free data structure can be found in \autoref{fig:prelim:lockfree}.

\begin{figure}[tp]
  \begin{cppcodeln}
    void push( const T &x ) {
        Node *t;
        Node *ltail = tail.load( std::memory_order_acquire );
        if ( ltail->write.load( std::memory_order_relaxed )
                == ltail->buffer + NodeSize )
            t = new Node();
        else
            t = ltail;

        *t->write.load( std::memory_order_relaxed ) = x;
        t->write.fetch_add( 1, std::memory_order_release );

        if ( ltail != t ) {
            ltail->next.store( t, std::memory_order_release );
            tail.store( t, std::memory_order_release );
        }
    }
  \end{cppcodeln}
  \caption{A push (enqueue) operation of a single-producer-single-consumer
    lock-free queue from older version of DIVINE (modified to use C++11 atomic
    operations).
    The queue uses a linked list of blocks (of type \texttt{Node}) which can
    hold a fixed number of elements of some type \texttt{T}.
    On lines 2--8 the function checks whether there is a room in the last
    allocated block and allocates a new block if necessary.
    Both \texttt{tail} and the \texttt{write} field in the \texttt{Node} are
    atomic variables and we explicitly use the memory ordering to avoid
    expensive synchronization on x86-64 processors.
    Then, on line 10, the actual value is written to the node, on the location
    pointed-to by the \texttt{write} field -- please note that the
    \texttt{write} pointer is atomic, but the actual value it points to is not
    atomic.
    On line 11 we shift the write pointer, which makes the data available to
    the consumer thread -- this is the point where we need to use release
    memory ordering to ensure safety on platforms which are more relaxed then
    x86 -- the release ordering ensures that any changes performed by the
    producer thread so far will be available to the consumer once it performs
    an acquire load on the \texttt{write} pointer (to check that the queue is
    not empty).
    Finally, on lines 13--16 we append the linked list of nodes if a new node
    was created on line 6.
    Again, we use the release memory ordering to ensure all operations
    performed so far (in particular the shift of \texttt{write} on line 11) are
    visible once these changes become visible.
    Please note that extending this approach to multiple producers is not
    trivial as it requires that the check if there is space available and the
    publication of the written value would need to be performed at the same
    time -- which would require either modification of two different locations
    in one atomic step (which is not possible on most processors and there is
    no C++ API for this operation) or complete redesign of the queue block.
  }\label{fig:prelim:lockfree}
\end{figure}

Overall, there is a steep compromise between code simplicity and performance.
A code which uses a few locks can be often reasonably understandable and the
programmer need not concert themself with relaxed memory as correct use of
locks makes its presence invisible, but performance might be degraded by
synchronization.
On the other hand, lock-free algorithms and data structures can have great
performance, but designing them and checking their correctness requires a lot
of effort and relaxed memory must be taken into account.

% \paragraph{Concurrent Data Structures}
% 
% Data structures which allow safe concurrent access are often used in
% multi-threaded programs for communication and sharing of state between threads.
% The advantage of their use is that often the communication can be restricted to
% be done only using these data structures and the rest of the code can then be
% designed more easily, without the need to consider concurrency.
% Furthermore, there are existing libraries of concurrent data structures which
% allow programmers to use high-performance and well-tested data structures
% without the need to invest to their development.
% Concurrent data structures are also interesting from the program analysis standpoint as verification of these libraries can impact 


\section{Programming Languages Used in this Thesis}

In this work, we often use example codes in figures.
We will mainly use C++ for examples of high-level code written by programmers,
and LLVM IR for example of low-level code more suitable for direct analysis.

\subsection{C++}

C++ is a high-level language well suited for wide variety of projects, from
code which directly interacts with hardware to GUI applications.
It can be high-performance and has good support for building of abstractions.
Since the C++11 version of the C++ standard, C++ also has native support for
threads and atomic variables with varying levels of atomicity guarantees.
Unless explicitly stated otherwise, all C++ examples in this work use the C++20
standard, as defined by its latest working draft N4860~\cite{cpp20}\mnote{The
N4860 draft should represent the final version of the standard which was
submitted to publication, but not yet published at the time of writing of this
paragraph.}.

\paragraph{Threads}
%
In C++ a thread is started by creation of an object of type \cpp{std::thread}.
\mnote{\cpp{std::} is a namespace which indicates this type belongs to the C++
standard library.}
Later, the thread can be waited-for by calling the \cpp{join} method of the
\cpp{std::thread} object.
Joining will block until the thread we are joining finishes.

\paragraph{High-Level Synchronization promitives}

C++ comes with various high-level synchronization primitives (which are usually
based on operating system level blocking primitives -- i.e., a thread waiting
on these synchronization primitives is suspended and not using any resources).
%
For the purposes of this work, we will use two synchronization primitives,
\emph{mutexes} and \emph{condition variables}.
Mutexes can be used to create mutual exclusion: only one thread can lock a
given mutex at any point.
Condition variables allow some threads to wait for signal from other threads.
They are often used in producer-consumer scenarios to signal to consumers that
there is some data ready for them.
Condition variables are always used together with mutexes.
They have two main operations: \texttt{wait} and \texttt{signal}: \texttt{wait}
is blocking call which suspend its calling thread until another thread calls
\texttt{signal}.
Waiting can only be ended if \texttt{signal} is called after the wait started,
the conditional variable has no way to detect that \texttt{signal} was called
before \texttt{wait}.
Furthermore, due to limitations of certain platforms, \texttt{wait} is allowed
to end spuriously (without being signalled).
For these two reasons, condition variables are usually used together with a
shared variable which indicates whether or not the consumer thread should
proceed; this variable should be guarded by the same mutex as the condition
variable used for signalling.

\paragraph{Atomic Variables and Low-Level Synchronization}
%
C++ has also support for atomic variables and atomic operations with them.
These atomic variables allow the program to take advantage of atomic hardware
instructions available on most platforms.
Atomic variables are mostly used in lock-free data structures and algorithms.

Atomic variables in C++ are variables of one of the atomic types:
\cpp{std::atomic_flag} and \cpp{std::atomic<T>} for some type \cpp{T} (which is
usually an integral of pointer type, for example \cpp{std::atomic<int>}).
For integral types, \cpp{std::atomic<T>} defines various operations which allow
atomic modification of the value of an atomic variable: for example, by calling
\cpp{fetch_add} on an atomic variable it is possible to atomically increment
its value and return its original value before the increment.
Some of these operations are also available with the operator syntax (e.g.,
using the operator \cpp{+=}).
For all types, it is possible to exchange the current value with a new one
(returning the previous value) and to preform compare-exchange (also sometimes
called compare-and-swap), which is a compound operation which atomically checks
that the value of an atomic variable is equal to a specified value and if it is
replaces the original value with a new one.

\subparagraph{Memory Ordering for Atomic Operations} \label{sec:prelim:cppmemord}

Without any additional settings all atomic instructions in C++ are sequentially
consistent i.e., there is a single global ordering of atomic operations on
which all threads agree.
However, C++ aims to allow programmers to fully utilize the performance of a
given platform and therefore it is possible to specify weaker constraints for
atomic operations.
These constraints are specified using so-called \emph{memory order}.
We will now shortly describe available memory orders (however, not describe
them precisely, as it is beyond scope and need of this work).

\begin{description}
    \item[\texttt{std::memory\_order::seq\_cst}] is the strongest and default
        memory order.
        It forces operations with this memory order to be sequentially
        consistent.

        \item[\texttt{std::memory\_order::release}] is memory order used with
            store operations (operations that write to the memory).
            It prevents other memory operations to be reordered after the
            release store.

        \item[\texttt{std::memory\_order::acquire}] is used with load operations
            (operations that read from the memory).
            It prevents other memory operations to be reordered before the
            acquire load.
            A release store and a subsequent acquire load from the same variable
            create a synchronization which ensures that all modification
            perfrormed in the storing thread before the release store are
            visible to the loading store after the corresponding acquire load.

        \item[\texttt{std::memory\_order::acq\_rel}] is used with atomic
            compound operations and combines the acquire and release orderings.

        \item[\texttt{std::memory\_order::consume}] is a weaker form of
            \texttt{acquire} which only affects data-dependent variables.
            It memory ordering is not widely used.

        \item[\texttt{std::memory\_order::relaxed}] is the weakest atomic
            ordering.
            It guarantees no synchronization and does not affect operations
            concerning other memory locations at all.
            It only guarantees atomicity of the given operations.
            Relaxed ordering can be used for example to implement atomic
            counters used for statistics.
\end{description}

\paragraph{Exceptions}

C++ has support for exceptions and is also able to specify that a given function
is not allowed to throw an exception.
In C++ it is possible to throw value of any type as an exception.
When an exception is thrown, it propagates to callers of the function which
have throw it until it triggers a \cpp{catch} code block which can catch it --
a \cpp{catch} block which either catches exactly the type of the exception, or
if the class of the exception uses inheritance, the exception can be also
caught a by a \cpp{catch} block which catches some of the predecessor types of
the exception.
Whenever the propagation of an exception causes end of a scope of some variable, its destructor is called to release resources associated with the variable, for example close an open file, release memory or release a mutex.

If a function is marked as \cpp{noexcept} and an exception would propagate from
it, the program terminates.


%%%%%%%%%%%%%%%%%%%%%%%%%%%%%%%%%%%%%%%%%%%%%%%%%%%%%%%%%%%%%%%%%%%%%%%%%%%%%%%%
\subsection{LLVM IR} %%%%%%%%%%%%%%%%%%%%%%%%%%%%%%%%%%%%%%%%%%%%%%%%%%%%%%%%%%%

LLVM is a compilation infrastructure which can be used to build optimizing
compilers.
The compiler build on LLVM consists of a language-specific frontend which
processes the source code and produces a language independent LLVM intermediate
representation, an optimizer which runs on the LLVM intermediate
representation, and a code generator which produces assembly for the given
platform.
LLVM intermediate representation (LLVM IR, LLVM code, or just LLVM), is a
low-level programming language which is mostly independent of both the
high-level language of the original program and the assembly language of the
given hardware platform.
Nevertheless, LLVM IR is somewhat influenced by the languages that are mainly
translated to it, namely C and C++.

LLVM IR is a type-safe assembly-like language.
Its basic operations are \emph{instructions} which take inputs of specific type
and produce output of (possibly different) type.
Values can be stored either in registers (each register is only assigned at one
place in the code -- the code is static single assignment) or in memory.
Memory can be further divided into global variables, which exist for the entire
run of the program, and dynamically allocated memory, which is obtained by a
call to an allocation function provided by the platform (i.e., the allocation
function is expected to be externally provided, memory allocation is not a part
of LLVM).

\paragraph{Memory Manipulation in LLVM}
Unlike for example the x86 machine code, most LLVM instructions do not modify
memory directly, but work with registers only.
Therefore, to change a value in memory, it is first necessary to load it, using
the \li{load} instruction, then modify it, and finally store it using the
\li{store} instruction.
There are two more instructions which can access memory, and these are used for
\emph{atomic compound operations}.
The \li{atomicrmw} instruction (atomic read-modify-write) can atomically
perform a load, arithmetic or logic operation, and a store or atomically
replace a value in the memory with another value, and in all cases it returns
the old value of the memory.
The \li{cmpxchg} (compare exchange) can atomically check if the value in memory
is the same as expected, and if so, replace it with a new value.
The atomic compound operations are often used to implement \emph{lock-free}
algorithms and data structures.
Atomic compound operations have memory order argument which specifies their
level of atomicity.
Similarly, \li{load} and \li{store} instructions can be atomic and then they
also come with a memory order argument.
Memory orders in LLVM are based on memory orders in C++ (see
\autoref{sec:prelim:cppmemord}).

\paragraph{Threads and High-Level Synchronization in LLVM}

LLVM has no primitives for starting and handling threads nor for high-level synchronization of threads in shared memory (mutexes, condition variables).
This matches well with the programming languages often translated to LLVM,
which usually implement threading using a library of thread-manipulation and
synchronization primitives.
Nevertheless, LLVM has a notion of thread-local variables; i.e., variables
which exist in a separate copy in each thread.

\paragraph{Memory Model}

Memory model of LLVM is mostly based on the memory model of C++, but instead of
atomic variables, it uses only atomic operations (i.e., it is theoretically
possible to combine atomic and non-atomic access to the same variable in LLVM).

\paragraph{Exceptions}

LLVM has support for exceptions.
Namely, there are two ways in which a function can be called in LLVM.
The \li{call} instruction is used for calls which either cannot throw an
exception, or which can throw an exception, but the exception does not need to
be inspected or caught in the functions which performs the \li{call}.
On the other hand, the \li{invoke} instruction is used if the exception needs
to be intercepted -- an \li{invoke} is a branching point.
If the function called by \li{invoke} returns normally the \cpp{invoke} behaves
like a \li{call} followed by a jump.
However, if an exception is propagated through \li{invoke} it transfers control
to a block of code which starts with a \li{landingpad} instruction which
recovers information about the exception.
The code which starts with the \li{landingpad} then decides how to handle the
exception -- it can be handled by this function, or a cleanup can be run and
the propagation of the exception can be resumed using the \li{resume}
instruction.

Interestingly, there is no instruction in LLVM to throw an exception.
Instead, the whole design assumes the platform for
which the code is compiled provides an exception support library -- so called
\emph{unwinder} which has also ability to throw the exception.
When the program is executing, the unwinder also takes care of the actual
propagation of the exception through the call stack (\emph{unwinding}).
For this purpose it needs metadata which is generated by the code generator
from the information in the \li{invoke}, \li{landingpad}, and \li{resume}
instructions.

\section{DIVINE}

DIVINE is an open-source verifier for C and C++ programs with focus on
concurrency and hard to discover bugs~\mcite{DIVINEToolPaper2017}.
DIVINE aims to be a general tool usable by programmers.
It can work with wide variety of programs and check for different property
violations, including assertion failures, memory access errors, memory leaks,
use of uninitialized memory and mutex locking errors.
There is also a limited support for liveness properties and an extension to
checking nontermination of parallel programs (\autoref{chap:lnterm},
\mcite{SB2019}).
Furthermore, DIVINE aims at full support of C and C++ including their standard
libraries -- currently we support C++17 with most of its standard library
(\autoref{chap:lang}, \mcite{SRB2017}).
DIVINE's platform model is loosely based on x86-64 Linux -- it uses 64
bit pointers and data types sizes and alignments are the same as used on this
platform.
With the optional support for the \xtso memory model (\autoref{chap:mm},
\mcite{SB2018x86tso}), DIVINE also respects the memory models of x86-64
processors.
Among the biggest differences between DIVINE and Linux on x86-64 are different calling conventions and stack layout; however, these differences should be transparent to a correct C/C++ program and programs which rely on platform-specific layout of the stack will be reported as buggy (the stack layout is not guaranteed by the C/C++ standards and therefore such programs are not well-defined).

\subsection{Architecture of DIVINE}

DIVINE is built on an explicit-state core called \divm which interprets LLVM
instructions and \divm extensions to LLVM IR called
\emph{hypercalls}~\mcite{RSCB2018}.
\divm hypercall behave as functions from the point of view of LLVM IR, but they
are treated as instructions by \divm.
These hypercalls handle operations like memory allocation and feeing and
nondeterministic choice.
They can also be used to switch stacks (for example to implement threading),
associate metadata with addresses, and to control the exploration of the state
space using interrupt points and cancellation.

In DIVINE the \divm has only limited knowledge of threads -- it can take their
existence into account for the purpose of state-space reductions, but cannot
start or terminate them and does not direct their scheduling.
Instead, scheduling is controlled by a scheduler which is linked to the
verified program and which uses the nondeterministic choice hypercall provided
by \divm.
Thread creation and termination is also implemented in a library.
Scheduling, thread management and a basic POSIX-compatible interface including
a filesystem model is grouped into \dios, a verification-oriented operating
system~\mcite{RBMKB2019}.

Finally DIVINE also uses program instrumentation to facilitate program
analysis.
Instrumentation adds interruption points to the program to specify at which
point the scheduler can be invoked -- these interruption points are used at
memory access instructions and at back-edges of loops (to make sure \divm does
not attempt run an infinite cycle).
Instrumentation is also used by the symbolic and abstract extensions of
DIVINE~\mcite{LRB2018}, for C++ exceptions (\autoref{chap:lang},
\mcite{SRB2017}), and if the program is analysed for relaxed memory behaviour
(\autoref{chap:mm}, \mcite{SB2018x86tso}).


\section{XXX:}

\subsection{Basic Program Features}

When it comes to the program analysis of programs written in high-level programming languages (such as C, C++, C\#, Java, Haskell),\mnote{Some readers might find it peculiar to consider C to be a high-level programming language; however, for our purposes the defining characteristic of a high-level language is that it is intended to be written by the programmer and allows them to use and define abstractions -- at the very least functions and data structures. In this way C is a high-level programming language, while assembly languages (such as x86-64 assembly) are low-level.} it is often necessary to choose a level on which the program will be analysed.
We can choose to analyse the high-level code directly -- in this case the analysis can be both precise and make use of high-level concepts (such as cycles or declarative description of certain thereading concepts e.g., using Open MP\mnote{\TODO{Open MP}}).
However, the direct analysis comes at high cost as programming languages often have rich and complex syntax and complicated semantics which is designed to aid programming, not verification.
For example, if we choose to handle in this way programs written in C++, we would need to concern ourselves with object hierarchies, exception handling, templates and many more.
This approach is also specific for the given programming language.


The on the other side of the spectrum, we can analyse a compiled program i.e., analyse machine code or assembly of given platform.
This approach can then be applied regardless of programming language and it works even on programs for which we have no source code available.
Nevertheless, there are serious disadvantages -- the analysis is required to support each hardware platform separately, and the machine code looses a lot of information present on the higher level.
For example, it is not possible to recover the bounds of stack variables from the x86 machine code directly (it might be recoverable using debugging information or exception-handling metadata, but these might not be available).

Finally, we can analyse some sort of intermediate or specialized representation, either a low-level language which still presents the information necessary for the analysis, or a specialized language designed for verification.
In either case, it is first necessary to translate the code in the programming language to the language used for verification and this can share many problems with the direct approach of verification of the high-level programming language.
However, if an existing intermediate representation accompanied with a translator from the high-level programming language is suitable for verification, this can side-step many of the problems described here.
Another advantage of using an intermediate representation is that the same representation can be used for multiple programming languages and multiple platforms -- this can make adaptation to new languages or platforms easier.
See \autoref{fig:prel:verif-levels}.

\begin{figure}
\center
\begin{tikzpicture}[->, >=stealth']

    \tikzstyle{box}=[rectangle, thick, draw]

    \begin{scope}
      \matrix[nodes=box, column sep = 0.5cm]{
        \node (c) {C}; &
        \node (cpp) {C++}; &
        \node (rust) {Rust}; &
        \node (hs) {Haskell}; &
        \node (cs) {C\#}; &
        \node (fs) {F\#}; &
        \node (java) {Java}; &
        \node (scala) {Scala}; \\
      };
    \end{scope}

    \begin{scope}[yshift=-2cm]
      \matrix[nodes={box, minimum width = 2cm}, column sep = 0.5cm]{
        \node (llvm) {LLVM IR}; &
        \node (net) {???}; &
        \node (jvm) {???}; \\
      };
    \end{scope}

    \begin{scope}[yshift=-4cm]
      \matrix[nodes={box, minimum width = 2cm}, column sep = 0.5cm]{
        \node (x86) {x86}; &
        \node (x86-64) {x86-64}; &
        \node (arm7) {ARMv7}; &
        \node (arm8) {ARMv8}; \\
      };
    \end{scope}

    \draw (c.south) edge (llvm)
          (cpp.south) edge (llvm)
          (rust.south) edge (llvm)
          (hs.south) edge (llvm);
    \draw (cs.south) edge (net)
          (fs.south) edge (net);
    \draw (java.south) edge (jvm)
          (scala.south) edge (jvm);

    \draw (llvm) edge (x86)
                 edge (x86-64)
                 edge (arm7)
                 edge (arm8);

    \draw (net) edge (x86)
                 edge (x86-64)
                 edge (arm7)
                 edge (arm8);

    \draw (jvm) edge (x86)
                 edge (x86-64)
                 edge (arm7)
                 edge (arm8);
%        edge [pre, draw=red] node[yshift=-1mm] {\texttt{0: lock(m)}} (i);

\end{tikzpicture}
\caption{}\label{fig:prel:verif-levels}
\end{figure}


% vim: colorcolumn=80 expandtab sw=4 ts=4 spell spelllang=en
